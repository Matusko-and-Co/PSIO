\documentclass[12pt,a4paper]{report}
\usepackage{amsmath}
\usepackage{amsfonts}
\usepackage{amssymb}
\usepackage{fullpage}
\usepackage[slovak]{babel}
\usepackage[utf8]{inputenc}
\usepackage[T1]{fontenc}
\usepackage{fullpage}
\usepackage{indentfirst}
\usepackage{array}
\usepackage{graphicx}
\usepackage{caption}
\DeclareGraphicsExtensions{.png,.jpg}

\begin{document}
\begin{titlepage}
\centering\bfseries
		Fakulta matematiky, fyziky a informatiky\\Univerzita Komenského v Bratislave	
	\vspace*{\stretch{2.0}}

	\fontsize{23}{28}\textbf{Špecifikácia požiadaviek na softvér}\\
	\vspace*{\stretch{0.05}}
	\fontsize{16}{22}\textbf{Predikcia šírenia infekčných ochorení}\\
	\vspace*{\stretch{0.2}}
	\large\textit{Matúš Čongrády\\Tibor Hanesz\\Jonatan Foltyn\\Katarína Šimnová}

	\vspace*{\stretch{2.0}}
\end{titlepage}\bigskip
	\setcounter{tocdepth}{9}
	\tableofcontents
	
\renewcommand{\chaptername}{}	
\chapter[Analýza technológií, dekompozícia a dátový model]{\rmfamily\bfseries
	Analýza technológií, dekompozícia a dátový model}
	

\section[Možné použité technológie a postupy]{\rmfamily\bfseries
	Možné použité technológie a postupy}

\subsection[Technológie]{\rmfamily\bfseries
	Technológie}
Na strane servera sme sa rozhodli použiť server nginx a PHP. Nginx preto, lebo nie je tak robustný, je stále pravidelne podporovaný a taktiež nezávislý od jedného operačného systému. PHP sme zvolili kvôli jednoduchosti, ľahkému nasadeniu, osobnej preferencie a skúsenosti a vzhľadom na jednoduchosť nie je potrebné hľadieť na rýchlosť do detailov. Na čítanie súborov z excelu použijeme voľne dostupnú knižnicu PHPExcel.
\par
Na strane klienta použije HTML5 a CSS3 na layout. Okrem toho použijeme JavaScript a knižnicu jQuery spolu s nadstavbou pre validáciu pre komfortnejšie užívateľské rozhranie. Tieto technológie sme zvolili vzhľadom na rozšírenosť, osobnú preferenciu a skúsenosť. 
\par
Na vykreslenie mapy použijeme Google Map API v3. Je to najrozšírenejšie maps API, ktoré nám ponúka presne tú funkcionalitu, ktorú potrebujeme.
\pagebreak

\section[Deployment diagram]{\rmfamily\bfseries
	Deployment diagram}
Prichádzajúce HTTP požiadavky vyhodnotí najprv nginx server a následne PHP. Posiela statický obsah ako HTML, CSS, JavaScript a obrázky. Taktiež posiela dáta na vytvorenie animácie prostredníctvom métody POST.
\begin{figure}[htb]
\includegraphics[scale=0.5]{deployment}
\caption[Deployment diagram]{Deployment diagram}
 \label{fig:Deployment diagram}
\end{figure}


\section[Domain model diagram]{\rmfamily\bfseries
	Domain model diagram}
Administrátor (zadávateľ) uploadne na stránku .xls súbor, ktorý server spracuje, a uloží si z neho dáta. Štruktúra uloženého údaju - počet ochorených / deň / kraj Slovenska. Server spracuje požiadavky od užívateľa a pošle statický obsah stránky. Webová stránka následne zobrazí animáciu pre užívateľa.
\begin{figure}[htb]
\includegraphics[scale=0.5]{Domain_model_diagram}
\caption[Domain model diagram]{Domain model diagram}
 \label{fig:Domain model diagram}
\end{figure}

\renewcommand{\chaptername}{}	
\chapter[Návrh]{\rmfamily\bfseries
	Návrh}

\section[Modelové triedy]{\rmfamily\bfseries
	Modelové triedy}

\subsection[Trieda Administration]{\rmfamily\bfseries
	Trieda Administration}
Atribúty:
\begin{itemize}
		\item FILE – Konštana obsahujúca názov súboru s údajmi. 
\end{itemize}
Metódy
\begin{itemize}
	\item changePassword(\$oldPassword, \$newPassword, \$newPasswordCheck) – Zmení heslo administrátora, ak je staré heslo správne a ak sú obe nové heslá rovnaké, inak vyhodí error.
	\item uploadFile(\$file) – Prepíše údaje z excel súboru do obyčajného súboru a zapíše ho na disk.
	\item loadData() – Prečíta údaje z uloženého súboru a pošle ich ako JSON object.
\end{itemize}

\subsection[Trieda Passwd]{\rmfamily\bfseries
	Trieda Passwd}
Atribúty:
\begin{itemize}
	\item SERCRET\_KEY – Konštanta, ktorá obsahuje tajný kľúč, ktorý sa pripíše ku heslo pri vytváraní hešu. 
	\item SALT – Konštanta, ktorá obsahuje kľúč, z ktorého sa vygeneruje soľ pre heš.
	\item FILE – Konštana obsahujúca názov súboru s heslom.
\end{itemize}
Metódy
\begin{itemize}
	\item readFromFile() – Prečíta heslo zo súboru.
	\item createNewPassword(\$passwd) – Zapíše nové heslo do súboru ako heš.
	\item makeHash(\$phrase) – Vytvorí z hesla heš.
	\item comparePasswords(\$passwd) – Porovná zadané heslo s heslom zo súboru.
\end{itemize}

\section[Triedy typu radič]{\rmfamily\bfseries
	Triedy typu radič}

\subsection[Trieda GetDataFromServer]{\rmfamily\bfseries
	Trieda GetDataFromServer}
Metódy
\begin{itemize}
	\item getDataFromServeromFile() – Získa údaje zo servera pomocou AJAXU a vytvorí z nich dvojrozmerné pole, ktoré vráti.
\end{itemize}

\end{document}