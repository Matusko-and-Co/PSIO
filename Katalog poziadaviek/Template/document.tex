\documentclass[12pt,a4paper]{report}
\usepackage{amsmath}
\usepackage{amsfonts}
\usepackage{amssymb}
\usepackage{fullpage}
\usepackage[slovak]{babel}
\usepackage[utf8]{inputenc}
\usepackage[T1]{fontenc}
\usepackage{fullpage}
\usepackage{indentfirst}
\usepackage{array}
\begin{document}
\begin{titlepage}
\centering\bfseries
		Fakulta matematiky, fyziky a informatiky\\Univerzita Komenského v Bratislave	
	\vspace*{\stretch{2.0}}

	\fontsize{23}{28}\textbf{Špecifikácia požiadaviek na softvér}\\
	\vspace*{\stretch{0.05}}
	\fontsize{16}{22}\textbf{Predikcia šírenia infekčných ochorení}\\
	\vspace*{\stretch{0.2}}
	\large\textit{Matúš Čongrády\\Tibor Hanesz\\Jonatan Foltyn\\Katarína Šimnová}

	\vspace*{\stretch{2.0}}
\end{titlepage}\bigskip
	\setcounter{tocdepth}{9}
	\tableofcontents
	
\renewcommand{\chaptername}{}	
	\chapter[Úvod]{\rmfamily\bfseries
Úvod}


\section[Predmet špecifikácie]{\rmfamily\bfseries
	Predmet špecifikácie}
Táto špecifikácia požiadaviek na softvér (ďalej ŠPS) popisuje používateľské, funkčné a parametrické požiadavky prvej verzie systému pre animácie šírenia infekčných ochorení. ŠPS je určená pre ľudí, ktorí budú budú implementovať. Špecifikácia je súčasťou zmluvy medzi objednávateľom a dodávateľom. Bude slúžiť ako východisko vyhodnocovania správnosti softvéru.

\section[Rozsah projetku a funkcie systému]{\rmfamily\bfseries
Rozsah projetku a funkcie systému}	
Tento projekt bude obsahovať animáciu šírenia infekcie v čase ako mapu Slovenska. Taktiež bude obsahovať možnosť ako zmeniť údaje o infekcii. Úlohou softvéru je služiť ako ukážka na rôznych prezentáciách. Systém predstavuje frontend a bude ho používať výhradne zadávateľ. 

\section[Slovník pojmov, Skratky]{\rmfamily\bfseries
	Slovník pojmov, Skratky}	
\begin{table}[h!]
	\centering
	\begin{tabular}{|>{\centering\arraybackslash}m{1in}|>{\centering\arraybackslash}m{1in}|}
		\hline
		\centering Pojem, skratka & vysvetlenie \\ [0ex]
		\hline
		Pojem, skratka & vysvetlenie \\ [0ex]
		\hline
	\end{tabular}
\end{table}
\section[Odkazy]{\rmfamily\bfseries
	Odkazy}

\section[Prehľad ďalšej časti]{\rmfamily\bfseries
	Prehľad ďalšej časti}


\renewcommand{\chaptername}{}	
\chapter[Celkový opis]{\rmfamily\bfseries
	Celkový opis}

\section[Kontent systému]{\rmfamily\bfseries
	Kontent systému}

\section[Funkcie systému]{\rmfamily\bfseries
	Funkcie systému}

\section[Triedy používateľov a ich vlastnosti]{\rmfamily\bfseries
	Triedy používateľov a ich vlastnosti}

\section[Odkazy]{\rmfamily\bfseries
	Odkazy}

\section[Obmedzenia]{\rmfamily\bfseries
	Obmedzenia}

\renewcommand{\chaptername}{}	
\chapter[Špecifikácia požiadaviek]{\rmfamily\bfseries
	Špecifikácia požiadaviek}

\renewcommand{\chaptername}{}	
\chapter[Prílohy]{\rmfamily\bfseries
	Prílohy}
\end{document}