\documentclass[12pt,a4paper]{report}
\usepackage{amsmath}
\usepackage{amsfonts}
\usepackage{amssymb}
\usepackage{fullpage}
\usepackage[slovak]{babel}
\usepackage[utf8]{inputenc}
\usepackage[T1]{fontenc}
\usepackage{fullpage}
\usepackage{indentfirst}
\usepackage{array}

\begin{document}
\begin{titlepage}
\centering\bfseries
		Fakulta matematiky, fyziky a informatiky\\Univerzita Komenského v Bratislave	
	\vspace*{\stretch{2.0}}

	\fontsize{23}{28}\textbf{Špecifikácia požiadaviek na softvér}\\
	\vspace*{\stretch{0.05}}
	\fontsize{16}{22}\textbf{Predikcia šírenia infekčných ochorení}\\
	\vspace*{\stretch{0.2}}
	\large\textit{Matúš Čongrády\\Tibor Hanesz\\Jonatan Foltyn\\Katarína Šimnová}

	\vspace*{\stretch{2.0}}
\end{titlepage}\bigskip
	\setcounter{tocdepth}{9}
	\tableofcontents
	
\renewcommand{\chaptername}{}	
	\chapter[Úvod]{\rmfamily\bfseries
Úvod}


\section[Predmet špecifikácie]{\rmfamily\bfseries
	Predmet špecifikácie}
Táto špecifikácia požiadaviek na softvér (ďalej ŠPS) popisuje používateľské, funkčné a parametrické požiadavky prvej verzie systému pre animácie šírenia infekčných ochorení. ŠPS je určená pre ľudí, ktorí ju budú výsledný software implementovať. Špecifikácia je súčasťou zmluvy medzi objednávateľom a dodávateľom. Bude slúžiť ako východisko vyhodnocovania správnosti softvéru.

\section[Rozsah projetku a funkcie systému]{\rmfamily\bfseries
Rozsah projetku a funkcie systému}
Výskyt niektorých infekčných chorôb je viazaný na určité oblasti (endemický výskyt). \\
Tento projekt bude obsahovať animáciu šírenia infekcie v určitej oblasti (krajoch Slovenska). Taktiež bude obsahovať možnosť zmeniť údaje o infekcii. Úlohou softvéru je služiť ako ukážka na rôznych prezentáciách pre lepšiu ilustráciu. Systém predstavuje frontend a bude ho používať predovšetkým zadávateľ. 

\section[Slovník pojmov, Skratky]{\rmfamily\bfseries
	Slovník pojmov, Skratky}	
\begin{table}[h!]
	\centering
	\begin{tabular}{|>{\centering\arraybackslash}m{1in}|>{\centering\arraybackslash}m{1in}|}
		\hline
		\centering Pojem, skratka & vysvetlenie \\ [0ex]
		\hline
		Predikcia & odhad vývoja, predpoveď \\ [0ex]
		\hline
		Infekčné ochorenie & nákazlivé ochorenie spôsobené mikroorganizmami \\ [0ex]
		\hline
	\end{tabular}
\end{table}
\section[Odkazy]{\rmfamily\bfseries
	Odkazy}

\section[Prehľad ďalšej časti]{\rmfamily\bfseries
	Prehľad ďalšej časti}


\renewcommand{\chaptername}{}	
\chapter[Celkový opis]{\rmfamily\bfseries
	Celkový opis}

\section[Kontext systému]{\rmfamily\bfseries
	Kontext systému}
	Cieľom je vytvoriť webovú aplikáciu slúžiacu pre zobrazenie priebehu šírenia infekčného ochorenia. Aplikácia má dopomôcť cieĺovému konzumentovi vytvoriť si vizuálnu predtavu širenia infekcie prostredníctvom animácie.

\section[Funkcie systému]{\rmfamily\bfseries
	Funkcie systému}
Užívateľ bude mať možnosť:
\begin{itemize}
	\item prehliadať mapu s predikciou šírenia infekčných chorôb	
	\item spustiť, prerušiť a zastaviť animáciu
	\item manuálne posunúť animáciu na konkrétne obdobie
Administrátor bude mať možnosť:
\begin{itemize}
	\item všetky možnosti, ktoré má užívateľ 
	\item prihlásiť sa pomocou svojho unikátneho administrátorského hesla
	\item načítať vstupný súbor s údajmi
	\item modifikovať predikčnú databázu
\end{itemize}

\section[Triedy používateľov a ich vlastnosti]{\rmfamily\bfseries
	Triedy používateľov a ich vlastnosti}
\begin{itemize}
	\item Užívateľ - je ľubovolná osoba, ktorá má záujem pozrieť si predikciu šírenia infekčných ochorení.
	\item Adminitrátor - je zadávateľ projektu alebo ním splnomocnená osoba. Má odbornú znalosť z problematiky predikcie šírenia infekčných ochorení. Je schopný
na základe svojich odborných výpočtov upraviť obsah stránky.

\section[Odkazy]{\rmfamily\bfseries
	Odkazy}

\section[Obmedzenia]{\rmfamily\bfseries
	Obmedzenia}
	Software nebude fungovať na webových prehliadačoch Internet Exploerer verzie 8 a starších. 

\renewcommand{\chaptername}{}	
\chapter[Špecifikácia požiadaviek]{\rmfamily\bfseries
	Špecifikácia požiadaviek}
\section[Popis prípadu použitia 1 - Prezeranie animácie]{\rmfamily\bfseries
	Popis prípadu použitia 1 - Prezeranie animácie}
\begin{table}[h!]
	\centering
	\begin{tabular}{|>{\centering\arraybackslash}m{1in}|>{\centering\arraybackslash}m{1in}|}
		\hline
		\centering Označenie & popis \\ [0ex]
		\hline
		Popis & Užívateľ navštívi hlavnú stránku, kde si spustí animáciu predikcie šírenia infekčných ochorení. Animácia sa mu
		zobrazí v grafickom okne ako mapa Slovenska rozdelená na kraje. V závislosti na percentuálnom výskyte ochorenia
		v populácii, sa daný okres zafarbí na farebnej škále od zelenej (malý výskyt) až po červenú (vysoký výskyt).\\ [0ex]
		\hline
		Opakovanosť & Niekoľkokrát za deň \\ [0ex]
		\hline
		Bežná cesta & 	\begin{itemize}
							\item Používateľ navštívi hlavnú stránku 
							\item Zvolí jednu z možností užívateľského rozhrania - "spustiť animáciu"
							\item Systém spustí animáciu v grafickom okne
							\item Užívateľ má ďalej možnost animáciu zastaviť, prerušiť a znovu spustiť
						\end{itemize} \\ [0ex]
		\hline
		Alternatívna cesta & - \\ [0ex]
		\hline
	\end{tabular}
\end{table}

\section[Popis prípadu použitia 2 - Zmena údajov databázy]{\rmfamily\bfseries
	Popis prípadu použitia 2 - Zmena údajov databázy}
\begin{table}[h!]
	\centering
	\begin{tabular}{|>{\centering\arraybackslash}m{1in}|>{\centering\arraybackslash}m{1in}|}
		\hline
		\centering Označenie & popis \\ [0ex]
		\hline
		Popis & Administrátor navštívi podstránku, ktorá nieje dostupná bežným užívateľom (
		užívateľ sa na ňu z hlavnej stránky nemá možnosť dostať žiadnym odkazom.) Administrátor
		zadá svoje prístupové heslo. Zobrazí sa mu možnosť načítať vstupný súbor. Tento
		súbor by mal obsahovať informácie (zadané ako matice) s dátami o širení infekčného
		ochorenia. Po úspešnom nahraní vstupného súboru sa aktualizujú zobrazované
		dáta na hlavnej stránke.\\ [0ex]
		\hline
		Opakovanosť & Niekoľkokrát za rok \\ [0ex]
		\hline
		Bežná cesta & 	\begin{itemize}
							\item Administrátor navštívi špeciálnu podstránku ktorá nieje dostupná bežným užívateľom
							\item Administrátor zadá svoje prístupové heslo
							\item Po správnom zadaní hesla mu systém umožní prístup k zmene údajov databázy
							\item Administrátor má možnosť nahrať na stránku súbor s údajmi o šírení infekčného ochorenia
							\item Po úspešnom nahraní súbor (súbor má správny formát) sa aktualizujú informácie na hlavnej stránke
						\end{itemize} \\ [0ex]
		\hline
		Alternatívna cesta & - \\ [0ex]
		\hline
		Výnimky & - \begin{itemize}
						\item Administrátor zadal nesprávne prístupové heslo - bude na to upozornený a dostane možnosť heslo zadať znova
						\item Súbor nahraný na stránku je v nesprávnom formáte - administrátor na to bude upozornený, a bude mať možnosť nahrať iný súbor
					\end{itemize} \\ [0ex]
		\hline
	\end{tabular}
\end{table}

\renewcommand{\chaptername}{}	
\chapter[Prílohy]{\rmfamily\bfseries
	Prílohy}
\end{document}